%
% Author: electricant <electricant@anche.no>
% Licensed under the GNU GPL v2
%
\documentclass{article}

% for hyperlinks
\usepackage{hyperref}
% for correct date
\usepackage[iso]{datetime}
% to easily separate things
\def\wl{\par \noindent \vspace{\lineskip}}

\title{Ultimate Machine}
\date {}
\author{The Electric Ant}
% disables chapter, section and subsection numbering
\setcounter{secnumdepth}{-1} 

\begin{document}
% Quick and dirty way to embed direct html code
\ScriptEnv{html}{\NoFonts\hfill\break}{\EndNoFonts}
\maketitle
\href{index.html}{Home}
\wl
\tableofcontents
%
% INTRODUCTION
%
\section{Introduction}
First of all you may be asking yourself what a ultimate machine is.
Basically it is a box with nothing on it besides a switch on one of its faces.
When you flip the switch a hand comes out from the box. The arm turns the switch off and goes back into the box.
\\*As \emph{Arthur C. Clarke} describes it:
\\*"There is something unspeakably sinister about a machine that does nothing - absolutely nothing - except switch itself off."
\\ So a frend of mine and I, mesmerized by the power of this gadget decided to build one.
There is plenty of instruction on the web (some of which is linked in the `Further information' section) but as you can imagine nothing we found applied to our use-case.
Some designs required two motors, others two switches and a complex assemby.
There were also some machines with a counter, some with light and also some whith sounds.
\\We got rid of all the bells and whistles of other designs and focused on functionality.
As a result the mechanical components required by our circuit are only one servo and a switch. The electronic components total to 14 (including LED and battery).
An added benefit of this circuit is that it draws no power (I really mean 0W!) when idling.
\\ Finally the circuit is based around a PIC10F220.
I have to aknlowledge that some people may not own such device, so a few alternatives are proposed. Howevever they are not yet tested (maybe  in the future they will) so try them at your own risk.
\\ If you happen to have any doubt/question/bug report \href{mailto:electricant@anche.no}{drop me a line}.
%
% CIRCUIT
%
\section{Circuit}
\begin{center}
\begin{html}
<a href="pictures/schem.JPG">
<img src="pictures/schem.JPG" alt="Schematic diagram" width=640></a>
\end{html}
\\Circuit schematic diagram
\end{center}
The circuit accepts any voltage from 3 to 5.5 volts.
Explaining how the circuit works is pretty straightforward.
The microcontroller controls the servo and keeps the circuit powered on for the time needed for the servo to operate.
\\When the user filps the switch a positive voltage is fed through the diode D1 into the base of Q1. Thransistor Q1 is now saturated and the base of Q2 is pulled to ground. Q2 turns on and powers the circuit and the servo. Now the control is passed to the PIC itself.
\\Diodes D1 and D2 form an OR gate which is connected to the switch and to GP2. Through GP2 the microcontroller keeps a positive voltage on the base of Q1 even when the switch is turned off. This feature is needed to move the servo bakwards after the machine has shut the switch off and place the arm in its rest position again.
\\The various resistors are there to prevent the PIC from damage in case a transistor gets broken.
\\No part in this circuit is critical. D1 and D2 can be any diode you have handy (as long as it works). Q1 can be any general purpose, small signal NPN transistor. Q2 any PNP transistor able to withstand at least 1A of continuous collector current.
(NOTE: R1 is calculated assuming $\beta_{Q1} \approx 100$. If a different transistor is used the new value is calculated according to the formula:
R1' = $(V_{batt} \cdot \beta)$ / $(I_{max})$; where $V_{batt}=4.5V$, $I_{max}=1.5A$).
%
% FIRMWARE
%
\section{Firmware}
The firmware and the instructions to build it are available on my GitHub profile (\href{https://github.com/electricant/ultimate_machine}{quick link}).
\\I won't dig into the code in detail (otherwhise why are there comments?). I'd rather explain which design choiches where made and why.
\\After calibrating and initializing itself the PIC goes through an infinite loop.
Its purpose is to generate reliable pulses for the servo.
To keep track of the elapsed time TMR0 is used.
Instead of busy-waiting for the timer to reach the appropriate value, the state of the switch is checked and the pulse-width is varied accordingly.
Moreover if the switch is OFF the pulses are counted in order to provide some delay prior to switching off the circuit and servo.
To program the board jus remove any conection from the reset jumper and close the program jumper insteda. Then connect the board to a PicKIT2/3 or anything else able to program the pic. Then fire your favuorite programmig tool.
\\(NOTE: the firmware is not completed. I'll keep updating this page if needed).

%
% PRINTED CIRCUIT BOARD
%
\section{Printed Circuit Board}
Only a few units are planned to be built an the circuit is pretty simple. So no PCB is provided. Maybe we'll add it in the future if required.
\\A piece of eurocard was used instead. There are two wiring diagrams:
\begin{center}
\begin{html}
<a href="pictures/rev3_pcb.JPG">
<img src="pictures/rev3_pcb.JPG" alt="REV3 PCB" width=640></a>
\end{html}
\\PCB with no SMD parts

\begin{html}
<a href="pictures/rev3_pcbSMT.JPG">
<img src="pictures/rev3_pcbSMT.JPG" alt="REV3 PCB SMT" width=640></a>
\end{html}
\\PCB with SMD parts
\end{center}
%
% FINAL RESULT
%
\section{Final Result}
The first attempt worked for a while and I was as happy as a pig in clover.
But happiness did not last long and a few switch-throws after the video below it stopped working. The circuit in fact remained always powered up.
Suspecting something was wired the wrong way a new board was designed from the ground up, just to discover that the transistor arrangement was flawed.
That is why the wiring diagram shows "REV 3.0" and the schematic is labeled "REV 2.0".
\\To date, only the circuit has been realized. Keep an eye on this website for updates.
\begin{center}
\begin{html}
<a href="pictures/rev1.JPG">
<img src="pictures/rev1.JPG" alt="REV1 image" width=640></a>
\end{html}
\\Picture of REV 1.0 with no PIC mounted

\begin{html}
<a href="pictures/rev3_front.JPG">
<img src="pictures/rev3_front.JPG" alt="REV3_front image" width=640></a>
\end{html}
\\REV 3.0 - Front view

\begin{html}
<a href="pictures/rev3_back.JPG">
<img src="pictures/rev3_back.JPG" alt="REV3_back image" width=640></a>
\end{html}
\\REV 3.0 - Back view. A mix of SMT and surface mount was used.

\begin{html}
      <video width="640" height="360" controls>
            <source src="video/ultimatema.ogg" type="video/ogg">
            Your browser does not support HTML5 video.
      </video>
\end{html}
\\Video of REV 1.0 in action. \href{video/ultimatema.ogg}{Download}.
\end{center}
%
% VARIANTS
%
\section{Variants}
This circuit can also accept two switches one for the user and one that works as a limit switch. Just remove the reset jumper and replace it with a switch connected as follows:
\\TODO: picture
\\As far as the firmware is concerned a patch will be published once the main project is complete.
\\You can also get rid of the PIC completely but another kind of circuit is needed and the limit switch becomes mandatory.
You'll also need to modify the servo by removing its control circuitry and hook the motor to terminals A and B of the cicuit.
Here is the schematic:
\\TODO: insert those damn pictures soon!
It is basically an H-Bridge operated by the switches. (If you don't know what an H-bridge is then google is your friend).
%
% FURTHER INFORMATION
%
\section{Further Information}
\begin{itemize}
\item \url{http://en.wikipedia.org/wiki/Claude_Shannon#Hobbies_and_inventions}
\item \url{http://www.youtube.com/watch?v=Z86V_ICUCD4}
\end{itemize}
\wl
\href{index.html}{Home}
\wl
Last edit: \emph{\today}
\end{document}
